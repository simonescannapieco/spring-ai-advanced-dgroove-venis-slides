\section{\faWrench\ Docker Model Runner} % (fold)
\label{sec:spring-ai-project-setup}
%
\begin{frame}[t,fragile] \frametitle{Docker Model Runner}
    \framesubtitle{Descrizione}
    \begin{itemize}[leftmargin=10pt,align=right]
        \onslide<1->\item[\alert{\faArrowCircleRight}] Risposta di Docker ad Ollama
        \onslide<2->
        \begin{itemize}[leftmargin=10pt,align=right]
            \item[\alert{\faArrowCircleRight}] LLM in Docker \textit{container} locali
            \item[\alert{\faArrowCircleRight}] Modelli AI generici dockerizzabili (WIP)
            \item[\alert{\faArrowCircleRight}] Docker mette a disposizione una serie di modelli \textit{open source} scaricabili tramite Engine o Desktop
        \end{itemize}

    \onslide<3->\item[\alert{\faExternalLink}] \url{https://www.docker.com/blog/run-llms-locally/}
    \item[\alert{\faExternalLink}] \url{https://www.docker.com/blog/introducing-docker-model-runner/}
    \end{itemize}
\end{frame}
%
\begin{frame}[t,fragile] \frametitle{Progetto Spring AI}
    \framesubtitle{Applicazione e passaggi}
    {\small
    \begin{itemize}[leftmargin=10pt,align=right]
        \item[\alert{\faArrowCircleRight}] Stub di progetto Spring AI multi-configurazione (Gemini + Ollama)
        \begin{itemize}[leftmargin=10pt,align=right]
            \onslide<2->\item[\alertedcircled{1}] Creazione \texttt{docker-compose.yml} per servizio Docker Ollama
            \onslide<3->\item[\alertedcircled{2}] Creazione \textit{file} variabili di ambiente per servizio Ollama
            \onslide<4->\item[\alertedcircled{3}] Creazione \textit{script} per \textit{start}, \textit{stop} ed eliminazione servizi Docker 
            \onslide<5->\item[\alertedcircled{4}] Creazione configurazione multi-LLM
            \onslide<6->\item[\alertedcircled{5}] Creazione modelli per domanda e risposta
            \onslide<7->\item[\alertedcircled{6}] Creazione interfaccia ed implementazione del servizio di richiesta
            \onslide<8->\item[\alertedcircled{7}] Creazione del controllore MVC
            \onslide<9->\item[\alertedcircled{8}] \textit{Test} delle funzionalità con Postman/Insomnia 
        \end{itemize}
    \end{itemize}
    }
\end{frame}
%
\begin{frame}[t,fragile] \frametitle{Progetto Spring AI}
    \framesubtitle{Servizio Docker Ollama - I}
        \begin{block}{\textit{File} \texttt{docker-compose.yml}}
			{\tiny\inputminted{yaml}{code/docker-compose.yml}}
    	\end{block}
\end{frame}
%
\begin{frame}[t,fragile] \frametitle{Progetto Spring AI}
    \framesubtitle{Servizio Docker Ollama - II}
        \begin{block}{\textit{File} \texttt{docker-compose.yml}}
			{\tiny\inputminted{yaml}{code/docker-compose-2.yml}}
    	\end{block}
\end{frame}
%
\begin{frame}[t,fragile] \frametitle{Progetto Spring AI}
    \framesubtitle{Variabili di ambiente servizio Docker Ollama}
        \begin{block}{\textit{File} \texttt{spring-ai.env}}
			{\tiny\inputminted{text}{code/spring-ai.env}}
    	\end{block}
\end{frame}
%
\begin{frame}[t,fragile] \frametitle{Progetto Spring AI}
    \framesubtitle{Script servizi Docker}
        \begin{block}{\textit{File} \texttt{start\_spring\_ai\_services.sh}}
			{\tiny\inputminted{bash}{code/start_spring_ai_services.sh}}
    	\end{block}

\end{frame}
%
\begin{frame}[t,fragile] \frametitle{Progetto Spring AI}
    \framesubtitle{Script servizi Docker}
        \begin{block}{\textit{File} \texttt{stop\_spring\_ai\_services.sh}}
			{\tiny\inputminted{bash}{code/stop_spring_ai_services.sh}}
    	\end{block}
\end{frame}
%
\begin{frame}[t,fragile] \frametitle{Progetto Spring AI}
    \framesubtitle{Script servizi Docker}
        \begin{block}{\textit{File} \texttt{erase\_spring\_ai\_services.sh}}
			{\tiny\inputminted{bash}{code/erase_spring_ai_services.sh}}
    	\end{block}

\end{frame}
%
\begin{frame}[t,fragile] \frametitle{Progetto Spring AI}
    \framesubtitle{Configurazione multi LLM}
        \vspace*{-.7cm}
        \begin{block}{Configurazione Gemini + Ollama}
			{\tiny\inputminted{java}{code/ChatClientConfig.java}}
    	\end{block}

\end{frame}
%
\begin{frame}[t,fragile] \frametitle{Progetto Spring AI}
    \framesubtitle{Modelli per domande e risposte}
        \begin{block}{Modello per domanda}
			{\tiny\inputminted{java}{code/Question.java}}
    	\end{block}
        \begin{block}{Modello per risposta}
			{\tiny\inputminted{java}{code/Answer.java}}
    	\end{block}
\end{frame}
%
\begin{frame}[t,fragile] \frametitle{Progetto Spring AI}
    \framesubtitle{Servizio multi LLM}
        \begin{block}{Interfaccia servizio}
			{\tiny\inputminted{java}{code/QuestionService.java}}
    	\end{block}
\end{frame}
%
\begin{frame}[t,fragile] \frametitle{Progetto Spring AI}
    \framesubtitle{Servizio multi LLM}
        \vspace*{-.7cm}
        \begin{block}{Implementazione servizio}
			{\tiny\inputminted{java}{code/QuestionServiceImpl.java}}
    	\end{block}
\end{frame}
%
\begin{frame}[t,fragile] \frametitle{Progetto Spring AI}
    \framesubtitle{MVC del servizio multi LLM}
    	\vspace*{-.7cm}
        \begin{block}{Implementazione controllore REST}
			{\tiny\inputminted{java}{code/QuestionController.java}}
    	\end{block}
\end{frame}
%
\begin{frame}[fragile] \frametitle{Codice}
    \framesubtitle{Branch di riferimento}
	\begin{center}
		{\scriptsize \href{https://github.com/simonescannapieco/spring-ai-advanced-dgroove-venis-code.git}{\texttt{https://github.com/simonescannapieco/spring-ai-base-dgroove-venis-code.git}}}\\
		\textit{Branch:} \alert{\texttt{1-spring-ai-gemini-ollama-configuration}}
	\end{center}
\end{frame}