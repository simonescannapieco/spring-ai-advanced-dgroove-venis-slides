\section{\faWrench\ Tool calling\\\small{Function as Tool}} % (fold)
\label{sec:spring-tool-calling-fat}
%
\begin{frame}[t,fragile] \frametitle{Progetto Spring AI}
    \framesubtitle{Applicazione e passaggi}
    {\small
    \begin{itemize}[leftmargin=10pt,align=right]
        \onslide<1->\item[\alert{\faArrowCircleRight}] Servizio FAT programmatico per richiesta temperatura (\textit{mock})
        \begin{itemize}[leftmargin=10pt,align=right]
            \onslide<2->\item[\alertedcircled{1}] Definizione modelli ed enumeratori per \texttt{WeatherRequest}, \texttt{TemperatureResponse} e \texttt{Unit}
            \onslide<3->\item[\alertedcircled{1}] Creazione \textit{tool} per richiesta temperatura
            \onslide<4->\item[\alertedcircled{2}] Configurazione \textit{client} Ollama e Gemini per utilizzo \textit{tool}
            \onslide<5->\item[\alertedcircled{3}] Creazione interfaccia e implementazione servizio
            \onslide<6->\item[\alertedcircled{4}] \textit{Test} delle funzionalità con Postman/Insomnia
        \end{itemize}
    \end{itemize}
    }
\end{frame}
%
\begin{frame}[t,fragile] \frametitle{Progetto Spring AI}
    \framesubtitle{Tool calling}
        \vspace*{-.7cm}
        \begin{block}{Enumeratore per unità di temperatura}
			{\tiny\inputminted{java}{code/Unit.java}}
    	\end{block}
        \begin{block}{Modello per inoltro dati per tempo atmosferico}
			{\tiny\inputminted{java}{code/WeatherRequest.java}}
    	\end{block}
        \begin{block}{Modello per risposta da tempo atmosferico}
			{\tiny\inputminted{java}{code/TemperatureResponse.java}}
    	\end{block}
\end{frame}
%
\begin{frame}[t,fragile] \frametitle{Progetto Spring AI}
    \framesubtitle{Tool calling}
        \vspace*{-.7cm}
        \begin{block}{Funzione di richiesta temperatura}
			{\tiny\inputminted{java}{code/TemperatureService.java}}
    	\end{block}
\end{frame}
%
\begin{frame}[t,fragile] \frametitle{Progetto Spring AI}
    \framesubtitle{Tool calling}
        \vspace*{-.7cm}
        \begin{block}{Configurazione Ollama e Gemini per \textit{tool calling} - I}
			{\tiny\inputminted{java}{code/WeatherToolsConfig.java}}
    	\end{block}
\end{frame}
%
\begin{frame}[t,fragile] \frametitle{Progetto Spring AI}
    \framesubtitle{Tool calling}
        \vspace*{-.7cm}
        \begin{block}{Configurazione Ollama e Gemini per \textit{tool calling} - II}
			{\tiny\inputminted{java}{code/WeatherToolsConfig-2.java}}
    	\end{block}
\end{frame}
%
\begin{frame}[t,fragile] \frametitle{Progetto Spring AI}
    \framesubtitle{Tool calling}
    	\vspace*{-.7cm}
        \begin{block}{Implementazione controllore REST - I}
			{\tiny\inputminted{java}{code/QuestionController.java}}
    	\end{block}
\end{frame}
%
\begin{frame}[t,fragile] \frametitle{Progetto Spring AI}
    \framesubtitle{Tool calling}
    	\vspace*{-.7cm}
        \begin{block}{Implementazione controllore REST - II}
			{\tiny\inputminted{java}{code/QuestionController-2.java}}
    	\end{block}
\end{frame}
%
\begin{frame}[fragile] \frametitle{Codice}
    \framesubtitle{Branch di riferimento}
	\begin{center}
		{\scriptsize \href{https://github.com/simonescannapieco/spring-ai-advanced-dgroove-venis-code.git}{\texttt{https://github.com/simonescannapieco/spring-ai-advanced-dgroove-venis-code.git}}}\\
		\textit{Branch:} \alert{\texttt{13-spring-ai-gemini-ollama-weather-tools}}
	\end{center}
\end{frame}