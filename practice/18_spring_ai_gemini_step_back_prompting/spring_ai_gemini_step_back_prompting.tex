\section{\faWrench\ Step-back prompting} % (fold)
\label{sec:spring-ai-gemini-step-back-prompting}
%
\begin{frame}[t,fragile] \frametitle{Progetto Spring AI}
    \framesubtitle{Descrizione}
    {\small
        \begin{itemize}[leftmargin=10pt,align=right]
            \onslide<1->\item[\alertedcircled{1}] \textit{Step-back prompting}
            \begin{itemize}[leftmargin=10pt,align=right]
                \item[\alert{\faArrowCircleRight}] Strategia: attivare parti di conoscenza di \textit{background} prima di utilizzare il LLM per la \textit{task}
                \begin{itemize}[leftmargin=10pt,align=right]
                    \item[\alert{\faArrowCircleRight}] Primo \textit{prompt} per chiedere informazioni generali correlate alla \textit{task} finale
                    \item[\alert{\faArrowCircleRight}] Secondo \textit{prompt} per sfruttare tali informazioni come contesto specifico e risolvere la \textit{task} finale
                    \item[\alert{\faExclamationTriangle}] Focalizzare l'attenzione del LLM su \alert{parti della sua conoscenza} 
                \end{itemize}
                \onslide<2->\item[\alert{\faArrowCircleRight}] Vantaggi
                \begin{itemize}[leftmargin=10pt,align=right]
                    \item[\alert{\faArrowCircleRight}] Utente lavora molto poco sulla creazione del contesto
                    \item[\alert{\faArrowCircleRight}] Contesti generalmente più mirati (generati dallo stesso LLM)
                    \item[\alert{\faArrowCircleRight}] Tende a mitigare i \textit{bias} cognitivi del LLM
                \end{itemize}
                \onslide<3->\item[\alert{\faArrowCircleRight}] Svantaggi
                \begin{itemize}[leftmargin=10pt,align=right]
                    \item[\alert{\faArrowCircleRight}] Costoso (utilizzo di due meccanismi di \textit{prompting})
                \end{itemize}               
            \end{itemize}
        \end{itemize}
    }
\end{frame}
%
\begin{frame}[t,fragile] \frametitle{Progetto Spring AI}
    \framesubtitle{Applicazione e passaggi}
    {\small
        \begin{itemize}[leftmargin=10pt,align=right]
            \item[\alert{\faArrowCircleRight}] Sistema di generazione nuova ambientazione per artefatto e genere
            \begin{itemize}[leftmargin=10pt,align=right]
                \onslide<2->\item[\alertedcircled{1}] Modifica modello \texttt{Artifact.java} per inserimento caratteristica genere
                \onslide<3->\item[\alertedcircled{2}] Creazione \textit{string template} per ambientazioni chiave nuovo artefatto (\textit{system} e \textit{user})
                \onslide<4->\item[\alertedcircled{3}] Creazione \textit{string template} per generazione artefatto da lista di ambientazioni chiave
                \onslide<5->\item[\alertedcircled{4}] Modifiche ad interfaccia ed implementazione del servizio Gemini
                \onslide<6->\item[\alertedcircled{5}] Modifica del controllore MVC per servizio Gemini
                \onslide<7->\item[\alertedcircled{6}] \textit{Test} delle funzionalità con Postman/Insomnia 
            \end{itemize}
        \end{itemize}
    }
\end{frame}
%
\begin{frame}[t,fragile] \frametitle{Progetto Spring AI}
    \framesubtitle{Artefatto}
        \begin{block}{Modello per artefatto}
			{\tiny\input{code/Artifact.java}}
    	\end{block}
\end{frame}
%
\begin{frame}[t,fragile] \frametitle{Progetto Spring AI}
    \framesubtitle{Prompt per ambientazioni chiave}
        \begin{block}{\textit{File} \texttt{get-key-settings-for-artifact-system-prompt.st}}
			{\scriptsize\input{code/get-key-settings-for-artifact-system-prompt.st}}
    	\end{block}
        \vspace*{.3cm}
        \begin{block}{\textit{File} \texttt{get-key-settings-for-artifact-user-prompt.st}}
			{\scriptsize\input{code/get-key-settings-for-artifact-user-prompt.st}}
    	\end{block}
\end{frame}
%
\begin{frame}[t,fragile] \frametitle{Progetto Spring AI}
    \framesubtitle{Prompt per generazione artefatto da lista di ambientazioni}
        \begin{block}{\textit{File} \texttt{get-generated-artifact-prompt.st}}
			{\scriptsize\input{code/get-generated-artifact-prompt.st}}
    	\end{block}
\end{frame}
%
\begin{frame}[t,fragile] \frametitle{Progetto Spring AI}
    \framesubtitle{Servizio Gemini}
        \vspace*{-.7cm}
        \begin{block}{Interfaccia servizio Gemini}
{\tiny\input{code/GeminiFromClientService.java}}
    \end{block}
\end{frame}
%
\begin{frame}[t,fragile] \frametitle{Progetto Spring AI}
    \framesubtitle{Servizio Gemini}
		\vspace*{-.7cm}
        \begin{block}{Implementazione servizio Gemini - I}
            {\tiny\input{code/GeminiFromClientServiceImpl.java}}
    \end{block}
\end{frame}
%
\begin{frame}[t,fragile] \frametitle{Progetto Spring AI}
    \framesubtitle{Servizio Gemini}
        \vspace*{-.7cm}
        \begin{block}{Implementazione servizio Gemini - II}
            {\tiny\input{code/GeminiFromClientServiceImpl-2.java}}
    \end{block}
\end{frame}
%
\begin{frame}[t,fragile] \frametitle{Progetto Spring AI}
    \framesubtitle{Servizio Gemini}
        \vspace*{-.7cm}
        \begin{block}{Implementazione servizio Gemini - III}
            {\tiny\input{code/GeminiFromClientServiceImpl-3.java}}
    \end{block}
\end{frame}
%
\begin{frame}[t,fragile] \frametitle{Progetto Spring AI}
    \framesubtitle{MVC del servizio Gemini}
    	\vspace*{-.7cm}
        \begin{block}{Implementazione controllore REST}
			{\tiny\input{code/QuestionFromClientController.java}}
    	\end{block}
\end{frame}
%
\begin{frame}[t,fragile] \frametitle{Progetto Spring AI}
    \framesubtitle{\ldots e ora a voi}
    {\small
        \begin{itemize}[leftmargin=10pt,align=right]
            \onslide<1->\item[\alert{\faArrowCircleRight}] Sistema di generazione di \alert{caratterizzazioni personaggi per un nuovo artefatto} (tipo e genere)
            \begin{itemize}[leftmargin=10pt,align=right]
                \item[\alert{\faArrowCircleRight}] Personaggio con nome, cognome, professione e carattere
                \item[\alert{\faArrowCircleRight}] Numero di personaggi parametrizzabile da \textit{request}
                \item[\alert{\faArrowCircleRight}] REST API \texttt{/client/generate/character} o per i più temerari, lavorare direttamente su \texttt{/client/generate}  
            \end{itemize}
        \end{itemize}
    }
\end{frame}
%
\begin{frame}[fragile] \frametitle{Codice}
    \framesubtitle{Branch di riferimento}
	\begin{center}
		{\scriptsize \href{https://github.com/simonescannapieco/spring-ai-base-dgroove-venis-code.git}{\texttt{https://github.com/simonescannapieco/spring-ai-base-dgroove-venis-code.git}}}\\
		\textit{Branch:} \alert{\texttt{16-spring-ai-gemini-step-back-prompting}}
	\end{center}
\end{frame}